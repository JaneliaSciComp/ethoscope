\section{Introduction}

Sleep is an apparently ubiquitous biological phenomenon whose precise roles remains poorly understood.
It is characterised by the reversible adoption of a postural immobility accompanied an elevated arousal threshold. 
In addition, sleep is also defined by the existence of an homoeostatic regulation.
This latter point can be demonstrated by the observation that sleep-deprived animals sleep longer than control, which is often referred as ``sleep rebound''.
Therefore, in order to fully characterise sleep in an model organism, it is necessary to be able to perform sleep deprivation (SD).

SD protocols are, however, non trivial in so far as their protocol can induce to very strong stress responses.
This makes it very difficult to discriminate between the effects resulting from general stress response and those specifically linked to SD.

Pharmacological and mechanical SD represent two widespread paradigms.
Drugs such as caffeine have been used to deprive animals from sleep. 
However, their precise mechanisms of action is rarely fully understood.
In addition, the toxicity of the drug and its effect on the structure of sleep are confounding.

Alternatively, mechanical SD has been widely used. It involves either preventing  sleep by forcing the animal to constantly move (e.g. putting a rodent in a water container), or by mechanically stimulating an animal at periodic, or random, intervals. These two approaches can be considered as very stressful for the subject.
For this reason, great interest has been ported toward ``dynamic'' mechanical SD.
In this paradigm, the experimental animals are disturbed if and only if they show signs of sleep.
This type of feedback loop system allow to minimise the number of stimuli delivered, thus reducing unnecessary stress.
In addition, it permits specific disruption of certain aspect of the structure of sleep. 
For instance, this approach can be used to systematically shorten sleep episodes.

The fruit fly (\emph{Drosophila melanogaster}) has been an extremely valuable research to understand the genetic basis of multiple biological phenomenons.
The discovery that fruit flies display a sleep behaviour rendered this model extremely valuable for sleep research.
Although the effect of sleep deprivation on its physiology have been investigated, most studies have been limited by the impossibility to perform dynamic sleep deprivation.
It is therefore becoming increasingly important to design a rigorous and high throughput protocol to deprive fruit fly from sleep. This new development will ultimately result in a more complete understanding of the physiology of sleep.

\section{Research plan}

During my PhD, I will start by designing an objective and high-throughput dynamic SD protocol. 
Then, I will compare the biological effects of different SD paradigms.
Finally, I will study in depth the physiological consequences of SD.

\subsection{Sleep depriving machine}

In a first place, I will be improving a prototypical device capable of rotating on demand 65mm tubes with single flies (gg, unpublished). 
In the meantime, I will be developing software to track position of individual animals using video camera, and interface with the device. 
The tracking algorithm will need to be robust and versatile. 
Furthermore, fast implementation is required since low power devices, such as raspberry pis, will be used in real-time.
The resulting algorithm will be able to detect quiescent (immobile) behaviour and will be thoroughly validated against human annotated ground-truth data.

\subsection{Impact of SD on behaviour and physiology}

In a second time, I will characterise the possible differences between commonly used SD protocols and the developed method.
This will involve quantification of phenotypes such as SD induced-mortality and intensity of the sleep rebound (\emph{i.e. how much sleep is recovered after SD}). 
Both wild type and established sleep and stress pathway mutants will be used.


\subsection{Effect of SD on gene regulation}

Finally, the effect of SD on temporal gene regulation will be investigated by performing a transcriptional analysis (RNAseq) of sleep deprived animal. In order to draw new hypothesis on the effect of SD, the level of transcripts in different organs will be analysed separately. 
This could, for instance, help to untangle the consequences restricted to the central nervous system from more systemic responses.
In addition, the time course of the transcriptional response to SD will be investigated.
Particular attention will be ported to characterised sleep pathways, but also on immune genes and general stress response pathways.
